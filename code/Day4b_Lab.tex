% Options for packages loaded elsewhere
\PassOptionsToPackage{unicode}{hyperref}
\PassOptionsToPackage{hyphens}{url}
%
\documentclass[
]{article}
\usepackage{amsmath,amssymb}
\usepackage{lmodern}
\usepackage{iftex}
\ifPDFTeX
  \usepackage[T1]{fontenc}
  \usepackage[utf8]{inputenc}
  \usepackage{textcomp} % provide euro and other symbols
\else % if luatex or xetex
  \usepackage{unicode-math}
  \defaultfontfeatures{Scale=MatchLowercase}
  \defaultfontfeatures[\rmfamily]{Ligatures=TeX,Scale=1}
\fi
% Use upquote if available, for straight quotes in verbatim environments
\IfFileExists{upquote.sty}{\usepackage{upquote}}{}
\IfFileExists{microtype.sty}{% use microtype if available
  \usepackage[]{microtype}
  \UseMicrotypeSet[protrusion]{basicmath} % disable protrusion for tt fonts
}{}
\makeatletter
\@ifundefined{KOMAClassName}{% if non-KOMA class
  \IfFileExists{parskip.sty}{%
    \usepackage{parskip}
  }{% else
    \setlength{\parindent}{0pt}
    \setlength{\parskip}{6pt plus 2pt minus 1pt}}
}{% if KOMA class
  \KOMAoptions{parskip=half}}
\makeatother
\usepackage{xcolor}
\usepackage[margin=1in]{geometry}
\usepackage{color}
\usepackage{fancyvrb}
\newcommand{\VerbBar}{|}
\newcommand{\VERB}{\Verb[commandchars=\\\{\}]}
\DefineVerbatimEnvironment{Highlighting}{Verbatim}{commandchars=\\\{\}}
% Add ',fontsize=\small' for more characters per line
\usepackage{framed}
\definecolor{shadecolor}{RGB}{248,248,248}
\newenvironment{Shaded}{\begin{snugshade}}{\end{snugshade}}
\newcommand{\AlertTok}[1]{\textcolor[rgb]{0.94,0.16,0.16}{#1}}
\newcommand{\AnnotationTok}[1]{\textcolor[rgb]{0.56,0.35,0.01}{\textbf{\textit{#1}}}}
\newcommand{\AttributeTok}[1]{\textcolor[rgb]{0.77,0.63,0.00}{#1}}
\newcommand{\BaseNTok}[1]{\textcolor[rgb]{0.00,0.00,0.81}{#1}}
\newcommand{\BuiltInTok}[1]{#1}
\newcommand{\CharTok}[1]{\textcolor[rgb]{0.31,0.60,0.02}{#1}}
\newcommand{\CommentTok}[1]{\textcolor[rgb]{0.56,0.35,0.01}{\textit{#1}}}
\newcommand{\CommentVarTok}[1]{\textcolor[rgb]{0.56,0.35,0.01}{\textbf{\textit{#1}}}}
\newcommand{\ConstantTok}[1]{\textcolor[rgb]{0.00,0.00,0.00}{#1}}
\newcommand{\ControlFlowTok}[1]{\textcolor[rgb]{0.13,0.29,0.53}{\textbf{#1}}}
\newcommand{\DataTypeTok}[1]{\textcolor[rgb]{0.13,0.29,0.53}{#1}}
\newcommand{\DecValTok}[1]{\textcolor[rgb]{0.00,0.00,0.81}{#1}}
\newcommand{\DocumentationTok}[1]{\textcolor[rgb]{0.56,0.35,0.01}{\textbf{\textit{#1}}}}
\newcommand{\ErrorTok}[1]{\textcolor[rgb]{0.64,0.00,0.00}{\textbf{#1}}}
\newcommand{\ExtensionTok}[1]{#1}
\newcommand{\FloatTok}[1]{\textcolor[rgb]{0.00,0.00,0.81}{#1}}
\newcommand{\FunctionTok}[1]{\textcolor[rgb]{0.00,0.00,0.00}{#1}}
\newcommand{\ImportTok}[1]{#1}
\newcommand{\InformationTok}[1]{\textcolor[rgb]{0.56,0.35,0.01}{\textbf{\textit{#1}}}}
\newcommand{\KeywordTok}[1]{\textcolor[rgb]{0.13,0.29,0.53}{\textbf{#1}}}
\newcommand{\NormalTok}[1]{#1}
\newcommand{\OperatorTok}[1]{\textcolor[rgb]{0.81,0.36,0.00}{\textbf{#1}}}
\newcommand{\OtherTok}[1]{\textcolor[rgb]{0.56,0.35,0.01}{#1}}
\newcommand{\PreprocessorTok}[1]{\textcolor[rgb]{0.56,0.35,0.01}{\textit{#1}}}
\newcommand{\RegionMarkerTok}[1]{#1}
\newcommand{\SpecialCharTok}[1]{\textcolor[rgb]{0.00,0.00,0.00}{#1}}
\newcommand{\SpecialStringTok}[1]{\textcolor[rgb]{0.31,0.60,0.02}{#1}}
\newcommand{\StringTok}[1]{\textcolor[rgb]{0.31,0.60,0.02}{#1}}
\newcommand{\VariableTok}[1]{\textcolor[rgb]{0.00,0.00,0.00}{#1}}
\newcommand{\VerbatimStringTok}[1]{\textcolor[rgb]{0.31,0.60,0.02}{#1}}
\newcommand{\WarningTok}[1]{\textcolor[rgb]{0.56,0.35,0.01}{\textbf{\textit{#1}}}}
\usepackage{graphicx}
\makeatletter
\def\maxwidth{\ifdim\Gin@nat@width>\linewidth\linewidth\else\Gin@nat@width\fi}
\def\maxheight{\ifdim\Gin@nat@height>\textheight\textheight\else\Gin@nat@height\fi}
\makeatother
% Scale images if necessary, so that they will not overflow the page
% margins by default, and it is still possible to overwrite the defaults
% using explicit options in \includegraphics[width, height, ...]{}
\setkeys{Gin}{width=\maxwidth,height=\maxheight,keepaspectratio}
% Set default figure placement to htbp
\makeatletter
\def\fps@figure{htbp}
\makeatother
\setlength{\emergencystretch}{3em} % prevent overfull lines
\providecommand{\tightlist}{%
  \setlength{\itemsep}{0pt}\setlength{\parskip}{0pt}}
\setcounter{secnumdepth}{-\maxdimen} % remove section numbering
\ifLuaTeX
  \usepackage{selnolig}  % disable illegal ligatures
\fi
\IfFileExists{bookmark.sty}{\usepackage{bookmark}}{\usepackage{hyperref}}
\IfFileExists{xurl.sty}{\usepackage{xurl}}{} % add URL line breaks if available
\urlstyle{same} % disable monospaced font for URLs
\hypersetup{
  pdftitle={Class 4 RMarkdown Lab},
  pdfauthor={Amy Winter},
  hidelinks,
  pdfcreator={LaTeX via pandoc}}

\title{Class 4 RMarkdown Lab}
\author{Amy Winter}
\date{07 Sept 2022}

\begin{document}
\maketitle

\hypertarget{r-markdown}{%
\section{R Markdown}\label{r-markdown}}

This is an R Markdown document. Markdown is a simple formatting syntax
for authoring HTML, PDF, and MS Word documents. For more details on
using R Markdown see \url{http://rmarkdown.rstudio.com}.

The way you can create a file like this in RStudio is: File → New File →
R Markdown and then using the default or using a template.

\hypertarget{you-can-use-hashtags-to-create-headers.-the-more-hastage-the-more-indented-the-header}{%
\subsection{You can use hashtags to create headers. The more hastage the
more indented the
header}\label{you-can-use-hashtags-to-create-headers.-the-more-hastage-the-more-indented-the-header}}

Below is a code chunk that will set up our packages and data.

\begin{Shaded}
\begin{Highlighting}[]
\FunctionTok{library}\NormalTok{(readxl)}
\NormalTok{dtp }\OtherTok{\textless{}{-}} \FunctionTok{read\_excel}\NormalTok{(}\StringTok{"../data/DTP\_Coverage\_WHO.xlsx"}\NormalTok{, }\AttributeTok{sheet=}\StringTok{"Data"}\NormalTok{)}
\FunctionTok{head}\NormalTok{(dtp, }\AttributeTok{n=}\DecValTok{5}\NormalTok{)}
\end{Highlighting}
\end{Shaded}

\begin{verbatim}
## # A tibble: 5 x 11
##   GROUP CODE  NAME   YEAR ANTIGEN ANTIG~1 COVER~2 COVER~3 TARGE~4  DOSES COVER~5
##   <chr> <chr> <chr> <dbl> <chr>   <chr>   <chr>   <chr>     <dbl>  <dbl>   <dbl>
## 1 COUN~ AFG   Afgh~  2021 DTPCV1  DTP-co~ ADMIN   Admini~ 1823296 1.70e6    93.4
## 2 COUN~ AFG   Afgh~  2021 DTPCV3  DTP-co~ ADMIN   Admini~ 1823296 1.48e6    81.2
## 3 COUN~ AFG   Afgh~  2020 DTPCV3  DTP-co~ ADMIN   Admini~ 1780564 1.52e6    85.3
## 4 COUN~ AFG   Afgh~  2020 DTPCV1  DTP-co~ ADMIN   Admini~ 1780564 1.73e6    97.2
## 5 COUN~ AFG   Afgh~  2019 DTPCV1  DTP-co~ ADMIN   Admini~ 1733907 1.64e6    94.4
## # ... with abbreviated variable names 1: ANTIGEN_DESCRIPTION,
## #   2: COVERAGE_CATEGORY, 3: COVERAGE_CATEGORY_DESCRIPTION, 4: TARGET_NUMBER,
## #   5: COVERAGE
\end{verbatim}

When you click the \textbf{Knit} button a document will be generated
that includes both content as well as the output of any embedded R code
chunks within the document.

\hypertarget{exercise-1}{%
\subsection{Exercise 1}\label{exercise-1}}

Create a chunk of code below that loads the tidyverse package (hint: use
library function), but change the options so that I don't see the output
of the loaded package. library()

\hypertarget{exercise-2}{%
\subsection{Exercise 2}\label{exercise-2}}

Create a code chunk below that creates a new data frame called
\texttt{dtp1.nig.df} that only includes data from Nigeria (hint: NAME is
``Nigeria'') and DTP containing vaccine dose 1 (hint: ANTIGEN is
``DTPCV1'') and only WUENIC estimates (hint: COVERAGE\_CATEGORY is
``WUENIC''). Output the number of rows of data are in this new data
frame, \texttt{dtp1.nig.df}?

\hypertarget{exercise-3}{%
\subsection{Exercise 3}\label{exercise-3}}

Create a code chunk below that modifies the data frame such that it only
induces the following variables: NAME, YEAR, ANTIGEN,
COVERAGE\_CATEGORY, COVERAGE. Output the first 8 rows of this data
frame. Output the number of columns in this modified data frame

\hypertarget{exercise-4}{%
\subsection{Exercise 4}\label{exercise-4}}

Create a code chunk below that outputs the mean dtp dose 1 coverage from
Nigeria across all years.

\hypertarget{exercise-5}{%
\subsection{Exercise 5}\label{exercise-5}}

Use the code chunk below to plot DTP coverage over time. Basically you
just need to fill in the variable names for \texttt{x=} and \texttt{y=}.
Below the code chunk interpret the plot for me.

\hypertarget{exercise-6}{%
\subsection{Exercise 6}\label{exercise-6}}

Knit this file. Save the .pdf in output subdirectory of you RProject as
``Day4\_Lab\_RMarkdown.pdf''

\end{document}
